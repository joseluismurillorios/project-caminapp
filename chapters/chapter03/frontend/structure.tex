La interfaz de usuario juega un papel muy importante en el aumento de la usabilidad de una aplicación, dado que la IU ofrece al usuario una vista abstracta de todo el sistema, el éxito del sistema depende en gran medida de ello. Por lo tanto, el diseño de la interfaz de usuario debe tener la importancia adecuada en el proceso del ciclo de vida del diseño del sistema.
\vspace{0.8cm}

\subsubsection{Módulo de inicio de sesión}
Uno de los desafíos es cómo implementar un esquema de autenticación y autorización flexible, seguro y eficiente. Parece confuso diferenciar entre autenticación y autorización. De hecho, es muy simple.

\begin{itemize}
  \item Autenticación: se refiere a verificar `quién es usted', por lo que debe usar el nombre de usuario y la contraseña para la autenticación.

  \item Autorización: se refiere a lo que `puede hacer', por ejemplo, acceder, editar o eliminar datos a algunos documentos, esto sucede después de la verificación.
\end{itemize}

Firebase Authentication proporciona servicios de back-end, SDK fáciles de usar y bibliotecas de interfaz de usuario para autenticar a los usuarios en la aplicación. En el código ejemplo \ref{login} se muestran los métodos necesarios para el manejo de sesiones del proyecto.
\vspace{0.8cm}

\lstinputlisting[style=ES6, label=login, caption=Fragmento de código del manejo de sesión de usuario]{code/login.js}

Gracias a la simplicidad y efectividad de los servicios de Firebase, este proceso es utilizado por muchas aplicaciones y servicios web en la actualidad.
\vspace{0.8cm}

\subsubsection{Evaluación de la interfaz de usuario}
El objetivo del proyecto es señalar los problemas y desafíos que surgen del uso de una pantalla táctil como dispositivo de entrada en una aplicación web. Para lograr esto, se desarrolló un prototipo, basado en pautas teóricas. El prototipo fue evaluado en sesiones informales donde los usuarios lo probaron y hablaron sobre cómo lo percibieron. Se alentó a los sujetos a sugerir soluciones alternativas y criticar las soluciones que encontraron negativas.
\vspace{0.8cm}

El prototipo inicial sufrió muy pocos cambios, siendo principalmente por motivos de funcionalidad y no estéticos. Todos los usuarios que sometieron a prueba la aplicación acordaron que el diseño era fácil de entender y se podían acostumbrar a el rápidamente. El resultado de la evaluación demostró que la interfaz responsiva funciona adecuadamente.