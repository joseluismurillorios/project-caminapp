La capacidad de caminar se ha reconocido cada vez más como un factor importante para el desarrollo urbano sostenible que, sin embargo, rara vez se ha investigado en ciudades de rápida urbanización. La caminabilidad captura la proximidad entre los usos del suelo funcionalmente complementarios o la conectividad entre destinos desde la perspectiva de accesibilidad, que a menudo se asocia con factores como el ancho de la calle, la conexión de la calle, los cruces peatonales, el número de carriles, velocidades seguras, etc. Además de la perspectiva de accesibilidad, la capacidad de caminar también se puede definir como un ambiente agradable para caminar, que está influenciado por una serie de factores, como la limpieza de las calles, los cruces seguros, la sensación de seguridad, la apariencia de los árboles de la calle, la iluminación nocturna, etc.
\vspace{0.8cm}

Por tales motivos se debe contar con un sistema fácil de usar para usuarios de distintas áreas de la ciudad y a su vez el diseño debe de adaptarse tanto a diferentes tamaños de pantallas como a diferentes dispositivos móviles. Se requiere una base de datos que notifique en tiempo real los cambios en la información asi como un servidor que administre los permisos adecuados para su manipulación.