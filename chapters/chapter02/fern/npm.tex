NPM (Node Package Manager) es el administrador de paquetes predeterminado para Node.js. Paquete es un término utilizado por npm para denotar herramientas que los desarrolladores pueden usar para sus proyectos \cite{goalkicker-node}. Se instala en el sistema con la instalación de Node.js. Los paquetes y módulos necesarios en un proyecto Node se instalan utilizando \textit{npm}.
\vspace{0.8cm}

NPM consta de tres componentes:
\begin{enumerate}
  \item Sitio web
  \item Registro
  \item CLI
\end{enumerate}
\subsubsection{Sitio web}
El sitio web oficial de npm es https://www.npmjs.com/. Con este sitio web puede encontrar paquetes, ver documentación, compartir y publicar paquetes.
\subsubsection{Registro}
El registro npm es una gran base de datos que consta de más de medio millón de paquetes. Los desarrolladores descargan paquetes del registro npm y publican sus paquetes en el registro.
\subsubsection{CLI (interfaz de línea de comando)}
Esta es la línea de comando que ayuda a interactuar con el npm para instalar, actualizar y desinstalar paquetes y administrar dependencias.
\subsection{Comandos NPM}
Npm tiene muchos paquetes que puedes usar en una aplicación para que su desarrollo sea más rápido y eficiente. Instalar módulos usando NPM no representa un gran problema. Hay una sintaxis simple para instalar cualquier módulo Node.js:
\vspace{0.8cm}

% \begin{verbatim}
%   npm install nombre-del-paquete
%   ejemplo: npm install express
% \end{verbatim}
\begin{lstlisting}[language=HTML]
  npm install nombre-del-paquete
  ejemplo: npm install express
\end{lstlisting}
% \lstinputlisting[style=ES6, caption=Comandos para instalar paquete con NPM]{code/npm.txt}